%% 
%% All Figures from Meta-RL PID Control Manuscript
%% Extracted for Reference and Standalone Compilation
%%

\documentclass[a4paper,fleqn]{article}

\usepackage{graphicx}          % 图片支持
\usepackage{caption}           % caption支持
\usepackage{subcaption}        % 子图支持
\usepackage{float}             % 提供[H]选项
\usepackage{bm}                % 粗体数学符号支持
\usepackage{amsmath}           % 数学支持
\usepackage{geometry}
\geometry{
    left=2.5cm,    % 左边距
    right=2.5cm,   % 右边距
    top=1.5cm,     % 上边距
    bottom=1.5cm   % 下边距
}

\begin{document}

\title{All Figures from Meta-RL PID Control Research}
\author{Extracted from Manuscript}
\date{\today}
\maketitle

\section{Figure 1: Meta-LF-PID Network Architecture}

\begin{figure}[H]
    \centering
    \includegraphics[width=0.95\textwidth]{neutral_network.pdf}
    \caption{Meta-LF-PID Network Architecture. The hierarchical feedforward network consists of an input layer (10D robot features $\mathbf{f}$ including mass, DOF, inertia, link lengths, and friction), two encoder layers with LayerNorm and ReLU activations (256D: $\mathbf{h}_1, \mathbf{h}_2$), a hidden layer (128D: $\mathbf{h}_{hidden}$), and structured parameter heads that output bounded LF-PID initialization $\hat{\bm{\theta}} \in [0,1]^D$ via sigmoid activation ($\sigma$). Here $\hat{\bm{\theta}}$ contains only per-joint input scaling factors and Takagi--Sugeno (TS) consequent parameters, while membership partitions are shared and fixed. The network is trained on $N=232$ high-quality filtered virtual robot variants using loss $\mathcal{L}_{meta} = \frac{1}{N}\sum_{v=1}^{N} w_v \|\bm{\theta}_v^* - \hat{\bm{\theta}}_v\|_2^2$, achieving fast inference (0.8ms per robot) with $\sim 10^5$ trainable parameters. The hierarchical encoder design ($W_1: 10 \times 256$, $W_2: 256 \times 256$, $W_3: 256 \times 128$) enables effective feature extraction and deep refinement for cross-platform generalization. Source: Authors own work.}
    \label{fig:meta_pid_arch}
\end{figure}

\section{Figure 3: Base Robot Platforms}

\begin{figure}[H]
\centering
\begin{subfigure}{0.32\textwidth}
    \centering
    \includegraphics[width=\textwidth]{franka_panda_simulation.png}
    \caption{Franka Panda (9-DOF)}
    \label{fig:franka_base}
\end{subfigure}
\hfill
\begin{subfigure}{0.32\textwidth}
    \centering
    \includegraphics[width=\textwidth]{kuka_iiwa_simulation.png}
    \caption{KUKA LBR iiwa (7-DOF)}
    \label{fig:kuka_base}
\end{subfigure}
\hfill
\begin{subfigure}{0.32\textwidth}
    \centering
    \includegraphics[width=\textwidth]{laikago_quadruped_simulation.png}
    \caption{Laikago (12-DOF)}
    \label{fig:laikago_base}
\end{subfigure}
\caption{Three base robot platforms serving as \textbf{training data sources} for physics-based data augmentation in PyBullet simulation. (a) Franka Panda manipulator (9-DOF) with complex serial kinematics, (b) KUKA LBR iiwa redundant manipulator (7-DOF) offering enhanced dexterity, and (c) Laikago quadruped (12-DOF) with parallel leg structure. From these diverse platforms, we generate 303 virtual variants through systematic perturbation of physical parameters (mass, inertia, friction, damping), filtering to 232 high-quality samples for meta-learning training. \textbf{Cross-platform validation is conducted on Franka Panda and Laikago}, which exhibit the greatest morphological differences (serial manipulator vs. parallel-legged quadruped), providing rigorous testing of generalization capability. Source: Authors own work.}
\label{fig:base_robots}
\end{figure}

\section{Figure 4: Per-Joint Error Comparison}

\begin{figure}[H]
  \centering
  \includegraphics[width=0.85\textwidth]{per_joint_error_comparison.png}
  \caption{Cross-platform generalization: Per-joint tracking error comparison across two morphologically distinct robot platforms. (a) Franka Panda serial manipulator (9-DOF) achieves 16.6\% overall improvement with exceptional gains in high-load joints (J2: +80.4\%, from 12.36° to 2.42°), demonstrating highly effective adaptation to manipulation tasks with concentrated loads. (b) Laikago parallel quadruped (12-DOF) achieves 2.1\% overall improvement, with individual joint improvements (+3.3\% to +7.7\% in 6 joints) slightly outweighing minor degradations (-3.7\% to -10.1\% in 6 joints). The contrast between platforms reveals an important engineering insight: RL adaptation excels when meta-learning exhibits localized high-error joints, while providing minimal benefit when baseline performance is uniformly strong. This guides practitioners to selectively deploy RL only where cost-effective. Error bars indicate standard deviation. Source: Authors own work.}
  \label{fig:per_joint_error}
\end{figure}

\section{Figure 5: Comprehensive Tracking Performance}

\begin{figure}[H]
  \centering
  \includegraphics[width=0.80\textwidth]{Figure4_comprehensive_tracking_performance.png}
  \caption{Comprehensive tracking performance comparison on Franka Panda. (a) Actual tracking error time series showing 10.9\% improvement with RL adaptation—RL reduces tracking oscillations and peak errors by smoothing control responses, (b) Error distribution histograms demonstrating tighter error bounds with Meta-PID+RL exhibiting more concentrated distribution around lower error values, (c) Per-joint error comparison with dual-axis visualization—left axis shows mean absolute error bars (Pure Meta-PID in blue, Meta-PID+RL in orange), right axis overlays improvement percentage curve (green line with markers) revealing Joint 2 benefits most with 80.4\% improvement; all 9 joints show positive gains averaging 12.1\%, and (d) Cumulative distribution function (CDF) showing consistent improvement across all error percentiles with 50th percentile improving by +10.5\% and 90th percentile by +11.0\%. Source: Authors own work.}
  \label{fig:actual_tracking}
\end{figure}

\section{Figure 6: Robustness Evaluation}

\begin{figure}[H]
  \centering
  \includegraphics[width=0.85\textwidth]{disturbance_comparison_final.png}
  \caption{Robustness evaluation across five disturbance scenarios on Franka Panda (20 episodes per scenario, evaluated across 100 random seeds for stochastic validation). Subplots (a-c) show detailed results for representative seed 51 (near-median performance), while subplot (d) presents the complete statistical distribution across all 100 seeds (mean±std: 4.81±1.64\% average improvement). The method achieves universal improvements across all tested conditions, with exceptional performance under parameter uncertainties (+19.2\%, from 35.90° to 29.01°), demonstrating remarkable adaptability to model discrepancies. Consistent gains in no disturbance (+13.2\%), payload variations (+8.1\%), mixed disturbances (+6.4\%), and random forces (+2.9\%) validate the robustness of the hierarchical Meta-PID+RL approach. Average improvement: +10.0\%. Source: Authors own work.}
  \label{fig:robustness}
\end{figure}

\section{Figure 7: RL Training Dashboard}

\begin{figure}[H]
  \centering
  \includegraphics[width=.95\textwidth]{rl_training_dashboard.png}
  \caption{Comprehensive RL training dynamics monitoring dashboard for Franka Panda (9-DOF) over 1M timesteps using PPO algorithm with optimized hyperparameters. (a) Episode reward improves progressively, demonstrating effective learning. (b) Value function loss decreases logarithmically, indicating convergence. (c) Policy loss stabilizes, showing robust policy optimization. (d) Entropy decreases gradually, showing the transition from exploration to exploitation. (e) Explained variance increases, validating effective value learning. (f) Clip fraction remains in the healthy range (0.05-0.15), confirming appropriate PPO hyperparameters. (g) Learning rate stays constant at $1 \times 10^{-4}$. (h) Gradient norm decreases and stabilizes below 0.5, indicating training stability. Source: Authors own work.}
  \label{fig:rl_training}
\end{figure}

\section{Summary of All Figures}

\subsection{List of Figures}
\begin{enumerate}
    \item \textbf{neutral\_network.pdf} - Meta-PID Network Architecture (Source: Authors own work)
    \item \textbf{franka\_panda\_simulation.png} - Franka Panda robot platform (Source: Authors own work)
    \item \textbf{kuka\_iiwa\_simulation.png} - KUKA LBR iiwa robot platform (Source: Authors own work)
    \item \textbf{laikago\_quadruped\_simulation.png} - Laikago quadruped platform (Source: Authors own work)
    \item \textbf{per\_joint\_error\_comparison.png} - Cross-platform per-joint error analysis (Source: Authors own work)
    \item \textbf{Figure4\_comprehensive\_tracking\_performance.png} - Comprehensive tracking performance (Source: Authors own work)
    \item \textbf{disturbance\_comparison\_final.png} - Robustness evaluation across disturbances (Source: Authors own work)
    \item \textbf{rl\_training\_dashboard.png} - RL training dynamics monitoring (Source: Authors own work)
\end{enumerate}

\subsection{Figure Categories}

\subsubsection{Methodology (2 figures)}
\begin{itemize}
    \item Figure 2: Meta-PID network architecture
    \item Figure 3: Base robot platforms (3 subfigures)
\end{itemize}

\subsubsection{Experimental Results (4 figures)}
\begin{itemize}
    \item Figure 4: Per-joint error comparison
    \item Figure 5: Comprehensive tracking performance
    \item Figure 6: Robustness evaluation
    \item Figure 7: RL training dashboard
\end{itemize}

\subsection{Figure Sources Summary}
\begin{itemize}
    \item \textbf{Authors own work} (8 figures/subfigures): All methodology, experimental results, and analysis figures
\end{itemize}

\end{document}
