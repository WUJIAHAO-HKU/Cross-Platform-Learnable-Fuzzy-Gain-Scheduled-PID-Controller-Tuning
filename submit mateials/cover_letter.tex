\documentclass[11pt]{letter}
\usepackage[margin=1in]{geometry}
\usepackage{hyperref}

% Author information
\signature{Jiahao Wu\\Corresponding Author}
\address{The University of Hong Kong\\
Hong Kong, China\\
Email: wuj277970@gmail.com}

\begin{document}

\begin{letter}{%
Editor-in-Chief\\
Robotica\\
Cambridge University Press}

\opening{Dear Editor-in-Chief,}

We are pleased to submit our manuscript entitled \textbf{``Cross-Platform Learnable Fuzzy Gain-Scheduled Proportional-Integral-Derivative Tuning via Physics-Constrained Meta-Learning and Reinforcement Learning Adaptation''} for consideration as a Research Article in \textit{Robotica}.

Proportional-Integral-Derivative controllers remain widely deployed in robotics, yet tuning stable gains is time-consuming and platform-specific. Our work addresses this through a hierarchical framework combining physics-constrained virtual robot synthesis with meta-learning and reinforcement learning for automated, cross-platform controller tuning.

\textbf{Key Contributions:}

\begin{itemize}
\item \textbf{Learnable Fuzzy Gain-Scheduled PID:} Structured controller with shared fuzzy partitions and learned per-joint scaling/consequent parameters
\item \textbf{Physics-Constrained Synthesis:} Generate 232 physically valid training variants from 3 base platforms via bounded perturbations (mass ±10\%, inertia ±15\%, friction ±20\%)
\item \textbf{Hierarchical Meta-RL:} Zero-shot initialization from 10D robot features, followed by 10-minute lightweight RL adaptation
\item \textbf{Cross-Platform Validation:} Tested on 9-DOF manipulator and 12-DOF quadruped under five disturbance scenarios, achieving 80.4\% error reduction on challenging joints and 19.2\% improvement under parameter uncertainty
\item \textbf{Design Insight:} Identification of \textit{optimization ceiling effect} characterizing when RL refinement provides maximal benefit
\end{itemize}

This work aligns with \textit{Robotica}'s emphasis on sound theory with realistic applications, maintaining interpretability of classical PID while achieving cross-platform transferability. The method reduces tuning effort from 30-60 minutes per platform to 5-minute one-time training plus optional 10-minute adaptation—a 3-6× efficiency gain relevant to both researchers and industrial practitioners.

\textbf{Declarations:} Original work not under consideration elsewhere. No competing interests or external funding. All code and data publicly available upon acceptance (MIT License). Complete reproducibility documentation included. All required statements (Author Contributions, Financial Support, Competing Interests, Data Availability) provided in manuscript.

Thank you for considering our manuscript for publication in \textit{Robotica}. We look forward to your response.

\closing{Sincerely,}

\end{letter}

\end{document}

